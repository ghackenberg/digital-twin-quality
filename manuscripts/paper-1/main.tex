\documentclass[9pt,conference]{IEEEtran}

\usepackage[colorlinks,allcolors=blue]{hyperref}

\begin{document}

    \title{Quality assurance of digital twins -- An experience report in the automotive industry}
    \author{
        \IEEEauthorblockN{
            Georg Hackenberg
        }
        \IEEEauthorblockA{
            School of Engineering\\
            University of Applied Sciences Upper Austria\\
            4600 Wels, Upper Austria, Austria\\
            \href{mailto:georg.hackenberg@fh-wels.at}{georg.hackenberg@fh-wels.at}
        }
        \and
        \IEEEauthorblockN{
            Alican Tüzün
        }
        \IEEEauthorblockA{
            School of Engineering\\
            University of Applied Sciences Upper Austria\\
            4600 Wels, Upper Austria, Austria\\
            \href{mailto:alican.tuezuen@fh-wels.at}{alican.tuezuen@fh-wels.at}
        }
    }
    \maketitle

    \begin{abstract}
        Digital twins are becoming more and more important for the efficient and effective development and operation of cyber-physical systems.
        However, digital twins are only useful if they reflect the real-world system accurately enough, i.e.\ their quality is high enough.
        This claim entails the question, what the term quality in the context of digital twins actually means and how it can be measured.
        In this article, we present our experience with quality assurance of a digital twin for an assembly line in the automotive industry.
        We explain our preliminary definition of digital twin quality, which we derive from classical quality models for general software systems.
        Furthermore, we describe quality issues, which we were able to detect in a digital twin of an assembly line in the automotive industry.
        Finally, we draw conclusions about how to leverage from our experience in different contexts and how to generalize the underlying approaches.
    \end{abstract}

    \section{Introduction}\label{section:introduction}
<<<<<<< Updated upstream

    Cyber-physical system (CPS), is a conceptual architecture/approach which aims to conjoin the physical and virtual space. 
    The aim of the CPS is to integrate communication and computing capabilities to physical assets to monitor, coordinate and control the physical space (where the physical assets exist) from the virtual space (where the virtual models exist), while achieve seamless real-time connection between the spaces.
    Digital twin approach is on the other hand, is a pragmatic solution, which is the execution of the CPS.~\cite{TAO20193}
    
    First occurrence of "Digital Twin" term  was introduced by Dr.~Michael Grieves in 2003 at the University of Michigan while giving  Product Lifecycle Management(PLM) course.~\cite{article}
    His introduction to term was rather raw, but still introduced a pragmatic solution to uncoupled solutions of global manufacturers, which were working only with physical or virtual products.

    His approach had three dimensions, along with a physical entity, a virtual entity and a connection between these entities which constructed the main frame for the today's evaluating digital twins.~\cite{article}

    Physical entity exists in the physical space with real functionalities and performance. They are responsible to accomplish certain tasks within the physical space, to give certain outputs.
    Different levels of abstraction can be usable for a physical entity such as an electric motor in the 1/10 of an RC Car, bearings in a certain machine, solar panels above some construction, human or an entire world can be counted as an example of a physical entity.
    Virtual entity exists in the virtual space with 

    With the improvement 
=======
    Cyber-physical system (CPS), is a conceptial architecture/approach which aims to conjoin the physical and virtual space. 
    The aim of the CPS is to integrate communication and computing capabilities to physical assets to monitor, coordinate and control the physical space (where the physical assets exist)
    from the virtual space (where the virtual models exist), while achieve seamless real-time connection between the spaces.
    Digital twin approach is on the other hand, is a pragmatic solution, which is the execution of the CPS.\cite{TAO20193}
    \section{What is a digital twin?}
    First occurance of "Digital Twin" term  was introduced by Dr.Michael Grieves in 2003 at the University of Michigan while giving 
    Product Lifecycle Management(PLM) course.\cite{article} His introduction to term was rather raw, but still introduced a pragmatic solution to uncoupled solutions of global manufacturers
    ,which were working only with physical or virtual products.\newline
    His approach had three dimensions, along with a physical entity, a virtual entity and a connection between these entities which constructed the main frame for the todays evaluating  
    digital twins. \cite{article} \newline
    Physical entity exists in the physical space with real functionalities and performance. They are responsible to accomplish certain tasks within the physical space, to give certain outputs. 
    Different levels of abstraction can be usable for a physical entity such as an electric motor in the 1/10 of a RC Car, bearings in a certain machine, solar panels above some construction, human or a entire world can be counted as an example of a physical entity.
    Virtual entity exists in the virtual space with 

    Fiziksel alan dendiginde, icerisinde birden fazla entity bulunduran(ornegin insanlar, aletler, binalar, hardware etc) alandir. Yani bu alan icerisinde, sekil, buyukluk, renk, yapi, kutle, hiz yani fiziksel tum ozellikler bulunur. Her bir entity belirli bir alan kaplar ve fizik kanunlarina uymak zorundadir. Bulunmus oldugu ortam kesin degildir(uncertain). Fiziksel entityler, belirli kisitlamalar altinda(ornegin zaman, para, kalite vs), pratik fonksiyonlara sahiptir. Entitylerden gelen datalar, analog datalardir ve smooth olarak degisirler.
    Virtual space, bilgisayar teknolojileri, simulasyon ve modelleme, haberlesme teknolojileri, internet, cloud computing, IOT vb gibi teknolojiler sayesinde olusturulmus alandir. Fiziksel alandan gelen analog datalari, discrete digital sinyallere ceviren ve kolayca saklanip, islenip gosterilmesini mumkun kilan bir alandir. Virtual space icindeki entityler, fiziksel alandaki entitylerin sadece geometrik sekillerini degil, ayni zamanda fiziksel ozellikleri ve davranislarinida(behaviour ve rule modeling) icermektedir. Birden fazla fiziksel alandan gelen datayi kullanarak, ornegin sensor datalari, insanlar, onceden bilinen bilgiler(domain knowledge), virtual alan bize, fiziksel alanin simulation, prediction, optimization ve VVT imkanini saglar.
    DigitalTwin, hem virtual space icerisindeki digital entityi, ki bu entity fiziksel alanda bulunan entitynin kopyasidir, hemde fiziksel alandaki entityi bulundurmaktadir. Digital twin yaklasiminin amaci, bu ikisi arasindaki siniri yok etme, yani fiziksel entity ile digital entityi birlestirmektir.  
>>>>>>> Stashed changes
    
    Genel olarak DT su ozellikleri yuzunden tercih edilir.
    \begin{itemize}
        \item Productun nasil davrandigi, product olusturulmadan once  3D olarak geometrik ve fiziksel olarak gozlemlenebilir. Bu gozlemlenebilirlik, bir urunu daha hizli markete sokmamiza olanak saglar. 
        \item  DT de fiziksel ve digital entityler surekli baglanti halindedir, boylelikle optimal operation PLM boyunca saglanir.
        \item VE uzerinden yapilan testlerle, harcanacak is gucu azaltilabilir. 
        \item VEnin prediction ozelligi kullanilarak, predictive maintenance gibi uygulamalar kullanilabilir.
        \item  VE nin visualize edilebilmesinden dolayi, shareholderlar sistem uzerinde daha iyi gorus saglarlar. 
        \item Information technologilerini, machine learning, simulation, optimization, iot, cloud computing, big data vs gibi  calisabilecegi bir framework sunar. Yani hepsi bir arada bulunabilir.
    \end{itemize}

    DT yapmanin challangeleri olarak sunlari siralayabiliriz,
    \begin{itemize}
        \item DT yi uygulayacak insanlarin teknik yetenekleri ve bilgilerinin yeterli olmamasi, kisitli olmasi
        \item High fidelity modeling, high fidelity simulation, real-time data collection, in depth data collection, flexible data architecture, high fidelity microservices gibi yuksek teknolojilerin kullanilmasi
        \item DT yi supportlayacak librarylerin veya sistemlerin yeteri kadar olmamasi
        \item Standartlasma eksikligi
        \item Proper VVT framework 
        \item Cost ve cost management 
        \item Cyber Security, Intellectual Property Rights
        \item Young technology
    \end{itemize}


    %/TODO SHOULD WE EXPLAIN OUR DIGITAL TWIN HERE?
<<<<<<< Updated upstream
=======
    %/TODO PT SIZE?
    \subsection{Methodology}
    TODO
>>>>>>> Stashed changes

    \subsection{Research objectives}

    Find approach for assessing the quality of digital twins...

    \subsection{Research questions}

    What is quality of digital twins? What is good quality? What is bad quality? What is impact of good quality? What is impact of bad quality? What are typical quality issues? How to assess quality? ...

    \subsection{Scientific contributions}

    Literature review (see Section~\ref{section:literature}), case study (see Section~\ref{section:case})...
    
    \section{Literature review}
    \label{section:literature}

    Methodology (see Section~\ref{section:liteature_methodology}), result (see Section~\ref{section:liteature_result}), and summary (See Section~\ref{section:liteature_summary})...

    \subsection{Methodology}
    \label{section:liteature_methodology}

    How did we find existing literature?
    
    \begin{itemize}
        \item ("digital twin" OR "simulation model" OR "system model" OR "CAD model") AND (quality OR defect OR issue OR error)
        \item ("digital twin" OR "simulation model" OR "system model" OR "CAD model") AND (verification OR validation OR accreditation)
    \end{itemize}

    \subsection{Result}
    \label{section:liteature_result}

    What is existing literature?

    \subsection{Summary}
    \label{section:liteature_summary}

    What do we learn from existing literature?
    (Existing literature does at most partially solve our problem!)

    \section{Case study (FELICE)}
    \label{section:case}

    TODO (Explain limitations of the FELICE digital twin! It is not a complete digital twin according to literature! It includes this, it is missing this.)

    \subsection{Methodology}
    \label{section:case_methodology}

    TODO

    \subsection{Result}
    \label{section:case_result}

    TODO

    \subsection{Summary}
    \label{section:case_summary}

    TODO

    \section{Conclusion}
    \label{section:conclusion}
    TODO

    \section*{Acknowledgements}
    TODO

    \bibliography{main}
    \bibliographystyle{plain}

\end{document}