\documentclass[9pt,conference]{IEEEtran}

\usepackage[colorlinks,allcolors=blue]{hyperref}
\usepackage{listings}
\usepackage{graphicx}
\usepackage{amsfonts}
\usepackage{verbatim}
\graphicspath{{./images/}}

\begin{document}

    \title{Quality assurance of digital twins -- An experience report in the automotive industry}
    \author{
        \IEEEauthorblockN{
            Georg Hackenberg
        }
        \IEEEauthorblockA{
            School of Engineering\\
            University of Applied Sciences Upper Austria\\
            4600 Wels, Upper Austria, Austria\\
            \href{mailto:georg.hackenberg@fh-wels.at}{georg.hackenberg@fh-wels.at}
        }
        \and
        \IEEEauthorblockN{
            Alican Tüzün
        }
        \IEEEauthorblockA{
            School of Engineering\\
            University of Applied Sciences Upper Austria\\
            4600 Wels, Upper Austria, Austria\\
            \href{mailto:alican.tuezuen@fh-wels.at}{alican.tuezuen@fh-wels.at}
        }
    }
    \maketitle

    \begin{abstract}
        Digital twins are becoming more and more important for the efficient and effective development and operation of cyber-physical systems.
        However, digital twins are only useful if they reflect the real-world system accurately enough, i.e.\ their quality is high enough.
        This claim entails the question, of what the term quality in the context of digital twins means and how it can be measured.
        In this article, we present our experience with the quality assurance of a digital twin for an assembly line in the automotive industry.
        We explain our preliminary definition of digital twin quality, which we derive from classical quality models for general software systems.
        Furthermore, we describe quality issues, which we were able to detect in a digital twin of an assembly line in the automotive industry.
        Finally, we conclude how to leverage our experience in different contexts and how to generalize the underlying approaches.
    \end{abstract}

    \section{Introduction}\label{section:introduction}
    A popular but highly misunderstood notion with fuzzy boundaries is the `digital twin.' 
    The misunderstanding began with the Apollo 13 mission, which many authors claim, had a first digital twin The twin of Apollo 13 was merely a tangible replication of the spacecraft that has been utilized in the physical simulations. Even though the `digital' aspect was expected to be present, 
    it wasn't~\cite{GrievesApollo13}. The historic D-day map (also known as the Big Board) in Southwick House England, which was used simultaneously before and during operation, can be argued to be similar. 
    The board was a twin of the operation, and it included models of battalions and ships that reflected the actual locations of the corresponding formations. 
    Furthermore, synchronization was implemented with radio communication to give directives from the real system to control the operation flow and also to update the twin's state~\cite{AMRC}.

    To begin analyzing `the digital twin' and its quality, authors needed to come up with a concrete description that involved a great level of abstraction. 
    The definition of a digital twin given by the digital twin consortium, according to the authors, is the best one currently available: `A digital twin is a virtual representation of real-world entities and processes, synchronized at 
    a specified frequency and fidelity.'~\cite{digitaltwinconsortium2022}.

    What makes that definition the best? Because it provides the highest abstraction level among the definitions of digital twins that we discovered during our literature analysis. The following is a poor example of abstraction taken from the initial definition of a digital twin from NASA's roadmap from 2010: `A digital twin is an integrated multi-physics, multi-scale, probabilistic simulation of a 
    vehicle or system that uses the best available physical models, sensor updates, fleet history, etc., to mirror the life of its flying twin~\cite{NASA}.'
    The digital twins that we encounter nowadays don't typically fall under NASA's definition. We want to stress that NASA's definition of a  digital twin is accurate since digital twins are systems that are `context' centered. 
    This definition, therefore, is appropriate in the context of NASA's use case and is thus accurate, but the abstraction level is not sufficiently high to allow for a thorough understanding of digital twins.

    In addition, we would want to emphasize the significance of Grieves' 2006 description of digital twins in the PLM book,
    which traces the origins of the concept. Although it featured the same components as a digital twin at the time, it was known as an Information Mirroring Model (IM)~\cite{GrievesPLMBook}. 
    They decided to use NASA's Digital Twin rather than IM after co-authoring the article with Vickers.

    \begin{figure}
        \centering
        \includegraphics[width =0.45\textwidth]{GrievesInformationMirroringModel.png}
        \caption{Grieves Information Mirroring Model}
        \label{fig:GrievesInformationMirroringModel}
        \cite{GrievesPLMBook}
    \end{figure}

    \subsection*{Real System}

    The term `real system' in the context of digital twins refers to a system that is a tangible thing that has the characteristics of a system. Depending on the system's context, we can use black-box, gray-box, or white-box methodologies to study the system. 
    The usage of various artifacts by Grieves in a system he referred to as `real space' can be seen in Figure~\ref{fig:GrievesInformationMirroringModel}.

    \subsection*{Virtual System}
    Virtual refers to an existence that exists in a computer system as opposed to a physical system. Therefore, virtual systems are abstract systems, which are nothing but an abstraction of the real system.
    The term `abstract systems' refers to systems that have had their set of variables and functions `abstracted' so that they may be computed by a solver, which in this case solver could be either the human brain or a computer.
    Additionally, just because a system is an abstract system doesn't mean it lacks a physical body; rather, it is abstracted. 
    Virtual systems can't exist on their own since, like abstract systems, they require a medium to exist. Computer systems, a higher system of virtual systems, are an ideal medium for virtual systems.
    
    \subsection*{Connection Between the Real and Virtual System}
    If we step back and observe the consortium's definition in~\ref{section:introduction}, which we told is the best definition we could find, 
    it talks about synchronization, fidelity and some specified frequency. As can be guessed these terms are attributes of the real-virtual connection.
    Synchronization is used here instead of twinning, mirroring and connecting. 
    Real-time is a time value, which is most of the time a constant value and defines the update frequency of the state of the virtual system of interest relative to the system of interest.
    Fidelity is the degree of precision and accuracy of our digital twin, relative to the system of interest.
    These words are highly dependent on the context of the digital twin.
    \subsection{Research objective}
    Find an approach to assess the quality of Digital twins~\cite{Jones2020}

    \subsection{Research question}\label{section: Research Questions}
    What is the quality of digital twins?

    \subsection{Research methodology}
    TODO

    \section{State of the art on quality}
    In the Oxford English Dictionary, the word `quality' is described as a noun as well as an adjective. The Adjective form indicates, a high standard or excellence. 
    For example, the phrase `quality products' implies that quality adds a high standard state to the product. 
    There are several meanings for the noun, but the one that interests us in this situation is `the standard of something as measured against other things of a similar kind. \cite{OxfordDictionary}'
    This definition shows the relativity of quality.

    According to ISO 9000:2015, quality is the degree to which a set of inherent characteristics of an object fulfills requirements~\cite{ISO9000}. Several words need to be expanded, to understand this definition. 
    
    
    First, the standard also explains characteristics as distinguishing features such as physical, sensory, behavioral, temporal, ergonomic and functional. 
    Instead of an object, a system can be used for more generalization.
    As for the requirements, the standard also explains as a requirement is a need or expectation that is stated, generally implied or obligatory.~\cite{ISO9000}

    Thus, when these definitions are combined, we obtain the following: `Quality is a degree to which a set of inherent distinguishing features of a system fulfills a need or expectation that is stated, generally implied or obligatory'. 
    It should be highlighted that there are various quality definitions by so-called `Quality Gurus', but the authors decided to expand the standard definition of ISO 9000.

    \subsection*{Product quality and Management}
    A product is a tangible or intangible system that satisfies the needs or wants of a customer. It could be a physical system like a car, or an abstract system, like a digital twin. 
    Regarding product quality, there are two different aspects~\cite{GrievesPLMBook}.
    First, quality is an attribute of a product, which meets product specifications. Specifications are mostly defined by the supra-systems subsystems, such as by the stakeholders within the supra-system,  If the stakeholders' expectations, or needs, are met, we can consider high quality.
    The ability to execute to a specific usage standard is the second facet of product quality. Since this standardization is usually not controllable, unlike the first, the system's quality will be determined by how the implied or obligatory standards are fulfilled.

    Quality management is a management type, which has an interest in Quality, and quality assurance is one of the parts of quality management, 
    which focuses on providing fidelity that quality requirements will be fulfilled. This means that quality assurance is a subsystem of quality management.

    Quality management systems(QMS) are being used to plan and execute the organization's quality policies and goals. 
    Examples of such systems can be a standard like ISO 9001 which can provide a framework for an organization, or a data-driven methodology 
    such as Six Sigma to reduce defects and improve efficiency. These concepts are not new, for example, ISO 9001 standard has been established in 1987, 
    and is still being regularly updated~\cite{ISO9001DebunkingtheMyth}.

    \subsection*{Software quality}
    Software is a  virtual system, which has a set of instructions, data or programs used to perform a specific task. 
    All virtual systems, which we described above, need a medium to live on, hence these systems can be found in a computer or other electronic device.  
    Software quality according to IEEE 730-2014, is: 
    `The degree to which a software product meets established requirements; however, 
    quality depends upon the degree to which those established requirements accurately represent stakeholder needs, wants, and expectations'~\cite{IEE730-2014}. 
    As can be seen, it's a generic explanation of a quality, which doesn't differ much from the explanation we gave above. 

    Software quality management techniques come from already used manufacturing techniques in the industries.
    For example, the quality assurance that we mentioned above, also has been used as a software quality management activity with some differences. 
    In manufacturing, products will never exactly meet a specification, due to errors in the machining so there should be some tolerance for the manufacturing. 
    But in the software systems, or more precisely in the virtual systems, it's most of the time, not the case. 
    Also its nearly impossible to conclude that, the software product fully meets its specifications\cite{SoftwareEngineering}.

    According to ISO/IEC:25010:2011, there are eight software quality attributes. Functional Suitability, 
    Performance efficiency, compatibility, usability, reliability, security, maintainability and portability.\cite{ISO/IEC:25010}
    Of course, not all aspects can be inspected and the person, who is responsible for the quality of the product, should choose the critical aspects of the software product and further the process according to that decision.

    \subsection*{Simulation Quality}
    Simulation models are not merely virtual systems; they are also abstracted systems, which are nothing more than imitations or, to be more precise, abstractions of real systems. 
    We are simulating every day when we are trying to predict the future outcome or abstractly thinking about the past state of some occurred event. 
    An important element to remember is that we mentally model the event before executing the simulation. However, most of the simulation processes we see today, 
    are computer-based, with software programs. These programs include ANSYS\cite{Ansys}, ABAQUS\cite{Abaqus}, and AnyLogic\cite{AnyLogic}, as examples. Consequently, the idea of simulation software programs emerges.

    Validation and verification can be used to evaluate the quality characteristics of the simulation model \cite{StewartSimulation} \cite{VerificationValidationSergent} \cite{OsmanBalci}. 
    Validation is the process of evaluating the simulation quality of the model in comparison to the real system from the perspective of the model's intended applications. 
    On the other hand, verification is a procedure to evaluate a simulation model's implementation and its associated data concerning the conceptual description and specifications of the model developer. 

    \subsection*{Digital twin quality}
    To analyze current DT literature within the concept of digital twin quality literature analysis was executed. 
    The analysis especially focused on how the digital twin quality is mentioned and assessed within the academic dissertations.

    \subsubsection*{Literature Search Methodology}
    Google Scholar (GS), Scopus (SCP) and Research Rabbit (RR) tools are used to fulfill our requirements for the literature review. Since there is a great number of digital twin publications in academics,
    narrowing the research is necessary, so the following algorithm with several levels of filtering will be applied.

    First filtering will be done with the following criteria

    \begin{itemize}
        \item  Dissertations should be written in English and published in a journal/conference.
        \item  Dissertations should not be older than 2016
        \item  Search term should be exactly searched
    \end{itemize}

    Filtering query calls for the GS and SCP are the following
    \begin{itemize}
        \item Google Scholar = `search term'
        \item Scopus = ALL (`digital twin quality') AND PUBYEAR > 2016 AND PUBYEAR < 2023 AND (LIMIT-TO (LANGUAGE, `English'))
    \end{itemize}
    Also want to note that, in the GS, there isn't a function that filters the dates, instead the user should click the Since 2016 option to filter the dates.

    Second filtering will be done to eliminate the duplications and even though authors already filtered the search for the English language, 
    still some results without the English language will appear. Again, language filtering should be done to clean the results.
    \begin{itemize}
        \item Remove Duplications
        \item Remove the dissertations which are not English manually
    \end{itemize}

    After the second filtration, the third filtering which is a manual investigation should occur with the following filtering criteria:
    \begin{itemize}
        \item  Proper literature review should be present in the selected dissertation
        \item  Abstract and conclusion should be relevant to the search term
    \end{itemize}

    If an interesting dissertation(s) would be found during the reading, it will be checked again with the first filter.

    \subsection*{Literature Result}
    First filtering was done for the search term "digital twin quality", meanwhile GS gave 41 different results, and SCP gave only 3 results. After the second filtering process, 3 duplications and 31 different results were found and 10 dissertations were not in English. 
    After the third filtering process, 9 dissertations passed and the rest is rejected.

    \subsection*{Important Findings}

    Important findings are the following;
    \begin{itemize}
        \item Shcherba et al. center their attention on the model quality, which is an observation for a digital twin subsystem. Despite being a crucial component of the digital twin, models alone cannot be used to assess the quality of the digital twin.
        \item A consistency evaluation approach for digital twin models was developed by He Zhang et al. and can be used to assess the quality of the models. Additionally, the authors discuss crucial ideas like consistency between real and virtual systems and ultra-fidelity models. The improvement of the service component is the primary emphasis of this article. Tao's five-dimensional digital twin concept includes this component.
        \item He Zhang et al. introduces the updating method for digital twin models in yet another insightful article. Once more, the accuracy and coherence of the model concepts were addressed.
        \item Selch et al. work is the only one that discusses the quality of the digital twin. The paper presents a machine-learning approach based on Bayesian networks to track the quality of the digital twin. The unique aspect of this study is that the author, rather than just analyzing the digital twin as a single system, determines the contributions of subsystems to forecast the overall quality of the digital twin. The digital couplings, which are linkages between the subsystems, are also mentioned by Selch et al. as a source of extra uncertainty. 
        Since only one digital twin has been used in practice, multiple subsystem digital twins have not been validated.
    \end{itemize}

    \section{Initial approach in the FELICE project}
    Inspired by ISO 25010...

    \subsection{First criterion: Correctness}
    \subsubsection{Definition}
    Correctness, 
    Car vs airplane (with nice pictures!)

    \subsubsection{Motivation}
    TODO

    \subsubsection{Findings from FELICE}
    TODO

    \subsection{Second criterion: Completeness}
    
    \subsubsection{Definition}
    Car vs engine (with nice pictures!)

    \subsubsection{Motivation}
    TODO

    \subsubsection{Findings from FELICE}
    TODO

    \subsection{Third criterion: Fidelity}
    
    \subsubsection{Definition}
    Linear vs differential (with nice pictures!)

    \subsubsection{Motivation}
    TODO

    \subsubsection{Findings from FELICE}
    TODO

    \subsection{Fourth criterion: Efficiency}
    
    \subsubsection{Definition}
    1ms vs 1min (with nice pictures!)

    \subsubsection{Motivation}
    TODO

    \subsubsection{Findings from FELICE}
    TODO

    \subsection{Fifth criterion: Evolvability}
    
    \subsubsection{Definition}
    Efforts to adapt the digital twin to new circumstances (e.g.\ add a new machine, replace the machine, etc.)

    \subsubsection{Motivation}
    TODO

    \subsubsection{Findings from FELICE}
    TODO

 

    \section{Future approach in the FELICE project}
    \label{section:framework_1}
    TODO

    \begin{figure}[htbp]
        \includegraphics[width=\columnwidth]{Digital Twin Deviation.png}
        \caption{TODO}
        \label{todo-2}
    \end{figure}

    \begin{figure}[htbp]
        \includegraphics[width=\columnwidth]{Digital Twin.png}
        \caption{TODO}
        \label{todo-1}
    \end{figure}

    \begin{figure}[htbp]
        \includegraphics[width=\columnwidth]{Continuous Quality Control.png}
        \caption{TODO}
        \label{todo-3}
    \end{figure}

   

    \section{Conclusion}~\label{section:conclusion}
    TODO

    \section*{Acknowledgements}
    TODO

    \bibliography{main}
    \bibliographystyle{plain}

\end{document}