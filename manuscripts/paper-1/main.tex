\documentclass[conference]{IEEEtran}

\usepackage[colorlinks,allcolors=blue]{hyperref}
\usepackage{amsfonts}

\begin{document}

    \title{Quality assurance of digital twins -- An experience report in the automotive industry}
    \author{
        \IEEEauthorblockN{
            Georg Hackenberg
        }
        \IEEEauthorblockA{
            School of Engineering\\
            University of Applied Sciences Upper Austria\\
            4600 Wels, Upper Austria, Austria\\
            \href{mailto:georg.hackenberg@fh-wels.at}{georg.hackenberg@fh-wels.at}
        }
        \and
        \IEEEauthorblockN{
            Alican Tüzün
        }
        \IEEEauthorblockA{
            School of Engineering\\
            University of Applied Sciences Upper Austria\\
            4600 Wels, Upper Austria, Austria\\
            \href{mailto:alican.tuezuen@fh-wels.at}{alican.tuezuen@fh-wels.at}
        }
    }
    \maketitle

    \begin{abstract}
        Digital twins are becoming more and more important for the efficient and effective development and operation of cyber-physical systems.
        However, digital twins are only useful if they reflect the real-world system accurately enough, i.e.\ their quality is high enough.
        This claim entails the question, what the term quality in the context of digital twins actually means and how it can be measured.
        In this article, we present our experience with quality assurance of a digital twin for an assembly line in the automotive industry.
        We explain our preliminary definition of digital twin quality, which we derive from classical quality models for general software systems.
        Furthermore, we describe quality issues, which we were able to detect in a digital twin of an assembly line in the automotive industry.
        Finally, we draw conclusions about how to leverage from our experience in different contexts and how to generalize the underlying approaches.
    \end{abstract}

    \section{Introduction}\label{section:introduction}

    Cyber-physical system (CPS), is a conceptual architecture/approach which aims to conjoin the physical and virtual space. 
    The aim of the CPS is to integrate communication and computing capabilities to physical assets to monitor, coordinate and control the physical space (where the physical assets exist) from the virtual space (where the virtual models exist), while achieve seamless real-time connection between the spaces.
    Digital twin approach is on the other hand, is a pragmatic solution, which is the execution of the CPS.~\cite{TAO20193}
    
    First occurrence of "Digital Twin" term  was introduced by Dr.~Michael Grieves in 2003 at the University of Michigan while giving  Product Lifecycle Management(PLM) course.~\cite{article}
    His introduction to term was rather raw, but still introduced a pragmatic solution to uncoupled solutions of global manufacturers, which were working only with physical or virtual products.

    His approach had three dimensions, along with a physical entity, a virtual entity and a connection between these entities which constructed the main frame for the today's evaluating digital twins.~\cite{article}

    Physical entity exists in the physical space with real functionalities and performance. They are responsible to accomplish certain tasks within the physical space, to give certain outputs.
    Different levels of abstraction can be usable for a physical entity such as an electric motor in the 1/10 of an RC Car, bearings in a certain machine, solar panels above some construction, human or an entire world can be counted as an example of a physical entity.
    Virtual entity exists in the virtual space with 

    With the improvement 
    
    %/TODO SHOULD WE EXPLAIN OUR DIGITAL TWIN HERE?

    \subsection{Research objectives}

    Find approach for assessing the quality of digital twins...

    \subsection{Research questions}

    What is a digital twin? What is quality of digital twins? What is good quality? What is bad quality? What is impact of good quality? What is impact of bad quality? What are typical quality issues? How to assess quality? ...

    \subsection{Scientific contributions}

    Literature review (see Section~\ref{section:literature}), case study (see Section~\ref{section:case})...
    
    \section{Literature review}
    \label{section:literature}

    Methodology (see Section~\ref{section:liteature_methodology}), result (see Section~\ref{section:liteature_result}), and summary (See Section~\ref{section:liteature_summary})...

    \subsection{Methodology}
    \label{section:liteature_methodology}

    How did we find existing literature?
    
    \begin{itemize}
        \item ("digital twin" OR "simulation model" OR "system model" OR "CAD model") AND (quality OR defect OR issue OR error)
        \item ("digital twin" OR "simulation model" OR "system model" OR "CAD model") AND (verification OR validation OR accreditation)
    \end{itemize}

    \subsection{Result}
    \label{section:liteature_result}

    What is existing literature?

    \subsection{Summary}
    \label{section:liteature_summary}

    What do we learn from existing literature?
    (Existing literature does at most partially solve our problem!)

    \section{Digital twin framework}
    \label{section:framework_1}

    TODO

    \subsection{Real system}

    TODO

    \subsubsection{Real state (RS)}

    TODO

    \subsubsection{Real state history (RSH)}

    TODO

    \[
        RSH: \mathbb{R} \rightarrow RS
    \]

    \subsubsection{Real state transition function (RSTF)}

    TODO

    \[
        RSTF: RS \rightarrow RS
    \]

    \subsubsection{Real state prediction (RSP)}

    TODO
    
    \subsection{Digital twin}

    TODO

    \subsubsection{Digital state (DS)}

    TODO

    \subsubsection{Digital state history (DSH)}

    In our digital world we cannot represent the continuous time domain of the real system completely.
    Instead, we have to discretize the time domain and sample the state history function.

    \[
        RSH: \mathbb{N} \rightarrow RS
    \]

    \subsubsection{Digital state transition function (DSTF)}

    Since we do not have perfect knowledge about real state and real state transition function, our simulation models typically are non-deterministic or probabilistic.
    Here we take a probabilistic viewpoint.

    \[
        DSRF: DS \rightarrow \mathbb{P}(DS)
    \]

    \subsubsection{Digital state prediction (DSP)}

    TODO

    \section{Digital twin quality framework}
    \label{section:framework_2}

    TODO

    \section{Case study (FELICE)}
    \label{section:case}

    TODO (Explain limitations of the FELICE digital twin! It is not a complete digital twin according to literature! It includes this, it is missing this.)

    \subsection{Methodology}
    \label{section:case_methodology}

    TODO

    \subsection{Result}
    \label{section:case_result}

    TODO

    \subsection{Summary}
    \label{section:case_summary}

    TODO

    \section{Conclusion}
    \label{section:conclusion}
    TODO

    \section*{Acknowledgements}
    TODO

    \bibliography{main}
    \bibliographystyle{plain}

\end{document}