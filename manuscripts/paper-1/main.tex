\documentclass[9pt,conference]{IEEEtran}

\usepackage[colorlinks,allcolors=blue]{hyperref}

\begin{document}

    \title{Quality assurance of digital twins -- An experience report in the automotive industry}
    \author{
        \IEEEauthorblockN{
            Georg Hackenberg
        }
        \IEEEauthorblockA{
            School of Engineering\\
            University of Applied Sciences Upper Austria\\
            4600 Wels, Upper Austria, Austria\\
            \href{mailto:georg.hackenberg@fh-wels.at}{georg.hackenberg@fh-wels.at}
        }
        \and
        \IEEEauthorblockN{
            Alican Tüzün
        }
        \IEEEauthorblockA{
            School of Engineering\\
            University of Applied Sciences Upper Austria\\
            4600 Wels, Upper Austria, Austria\\
            \href{mailto:alican.tuezuen@fh-wels.at}{alican.tuezuen@fh-wels.at}
        }
    }
    \maketitle

    \begin{abstract}
        Digital twins are becoming more and more important for the efficient and effective development and operation of cyber-physical systems.
        However, digital twins are only useful if they reflect the real-world system accurately enough, i.e.\ their quality is high enough.
        This claim entails the question, what the term quality in the context of digital twins actually means and how it can be measured.
        In this article, we present our experience with quality assurance of a digital twin for an assembly line in the automotive industry.
        We explain our preliminary definition of digital twin quality, which we derive from classical quality models for general software systems.
        Furthermore, we describe quality issues, which we were able to detect in a digital twin of an assembly line in the automotive industry.
        Finally, we draw conclusions about how to leverage from our experience in different contexts and how to generalize the underlying approaches.
    \end{abstract}

    \section{Introduction}\label{section:introduction}

    Cyber-physical system (CPS), is a conceptual architecture/approach which aims to conjoin the physical and virtual space tightly. 
    The aim of the CPS is to integrate communication and computing capabilities to physical entities to monitor, coordinate and control the physical space (where the physical assets exist) from the virtual space (where the virtual models exist), while achieving seamless, real-time connection between the spaces.
    Digital twin approach is on the other hand, is a pragmatic solution, which is the execution of the CPS idea.~\cite{TAO20193}
    
    First occurrence of "Digital Twin" term  was introduced by Dr.~Michael Grieves in 2003 at the University of Michigan while giving  Product Lifecycle Management(PLM) course.~\cite{article}
    His introduction to term was rather raw, but still introduced a pragmatic solution to gather more data from the physical entities, to achieve the "Lean" idea.
    
    His approach has three dimensions, along with a physical entity, a virtual entity and a connection between these entities which constructed the main frame for the today's evaluating digital twins.~\cite{article}

    Physical entity exists only in the physical space with real functionalities and performance. It is responsible to accomplish certain tasks within the physical space, to give certain outputs.
    Different levels of abstraction can be usable for a physical entity such as an electric motor in the 1/10 of an RC Car, bearings in a certain machine, solar panels above some construction, human or an entire world can be counted as an example of a physical entity.

    Virtual entity exists only in the virtual space, which should imitate the physical entity with high fidelity models. If we consider the virtual entity as a function;

    \begin{equation}\label{Formulated Entity Equation}
        VE = (G_v, P_v, B_v, R_v)
    \end{equation}
    where:
    
    $G_v$ represents the geometry model, $P_v$ represents the physical model, $B_v$ represents  the behavior model, and $R_v$ represents the rule model~\cite{article}.
    
    The last dimension, is the connection between the physical space and virtual space. Physical space created realtime data, and through the connection, these data will be received by the virtual space, 
    and updates the states of the models. Virtual space sends the needed information to the physical space. 

    There is also another, up to date approach which adds two more dimensions(Data and Services), to the traditional digital twin model which is descriebed above.~\cite{article}

    \begin{equation}
        M_{dt} = (PE,VE,Ss,DD,CN)
    \end{equation}
    \begin{description}
        \item[PE] Represents Physical Entity
        \item[VE] Represents Virtual Entity
        \item[Ss] Represents Digital Twin Services
        \item[DD] Represents Digital Twin Data
        \item[CN] Represents Connections
        \end{description}
    
    %/TODO SHOULD WE EXPLAIN OUR DIGITAL TWIN HERE?

    \subsection{Research objectives}

    Find approach for assessing the quality of digital twins...

    \subsection{Research questions}

    What is quality of digital twins? What is good quality? What is bad quality? What is impact of good quality? What is impact of bad quality? What are typical quality issues? How to assess quality? ...

    \subsection{Scientific contributions}

    Literature review (see Section~\ref{section:literature}), case study (see Section~\ref{section:case})...
    
    \section{Literature review}
    \label{section:literature}

    Methodology (see Section~\ref{section:liteature_methodology}), result (see Section~\ref{section:liteature_result}), and summary (See Section~\ref{section:liteature_summary})...

    \subsection{Methodology}
    \label{section:liteature_methodology}

    How did we find existing literature?
    
    \begin{itemize}
        \item ("digital twin" OR "simulation model" OR "system model" OR "CAD model") AND (quality OR defect OR issue OR error)
        \item ("digital twin" OR "simulation model" OR "system model" OR "CAD model") AND (verification OR validation OR accreditation)
    \end{itemize}

    \subsection{Result}
    \label{section:liteature_result}

    What is existing literature?

    \subsection{Summary}
    \label{section:liteature_summary}

    What do we learn from existing literature?
    (Existing literature does at most partially solve our problem!)

    \section{Case study (FELICE)}
    \label{section:case}

    TODO (Explain limitations of the FELICE digital twin! It is not a complete digital twin according to literature! It includes this, it is missing this.)

    \subsection{Methodology}
    \label{section:case_methodology}

    TODO

    \subsection{Result}
    \label{section:case_result}

    TODO

    \subsection{Summary}
    \label{section:case_summary}

    TODO

    \section{Conclusion}
    \label{section:conclusion}
    TODO

    \section*{Acknowledgements}
    TODO

    \bibliography{main}
    \bibliographystyle{plain}

\end{document}