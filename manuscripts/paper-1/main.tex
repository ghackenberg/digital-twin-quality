\documentclass[conference]{IEEEtran}

\usepackage[colorlinks,allcolors=blue]{hyperref}

\begin{document}

    \title{Quality assurance of digital twins -- An experience report in the automotive industry}
    \author{
        \IEEEauthorblockN{
            Georg Hackenberg
        }
        \IEEEauthorblockA{
            School of Engineering\\
            University of Applied Sciences Upper Austria\\
            4600 Wels, Upper Austria, Austria\\
            \href{mailto:georg.hackenberg@fh-wels.at}{georg.hackenberg@fh-wels.at}
        }
        \and
        \IEEEauthorblockN{
            Alican Tüzün
        }
        \IEEEauthorblockA{
            School of Engineering\\
            University of Applied Sciences Upper Austria\\
            4600 Wels, Upper Austria, Austria\\
            \href{mailto:alican.tuzuen@fh-wels.at}{alican.tuzuen@fh-wels.at}
        }
    }
    \maketitle

    \begin{abstract}
        Digital twins are becoming more and more important for the efficient and effective development and operation of cyber-physical systems.
        However, digital twins are only useful if they reflect the real-world system accurately enough, i.e.\ their quality is high enough.
        This claim entails the question, what the term quality in the context of digital twins actually means and how it can be measured.
        In this article, we present our experience with quality assurance of a digital twin for an assembly line in the automotive industry.
        We explain our preliminary definition of digital twin quality, which we derive from classical quality models for general software systems.
        Furthermore, we describe quality issues, which we were able to detect in a digital twin of an assembly line in the automotive industry.
        Finally, we draw conclusions about how to leverage from our experience in different contexts and how to generalize the underlying approaches.
    \end{abstract}

    \section{Introduction}
    \label{section:introduction}
    TODO~\cite{Boschert2016,Fuller2020,Jones2020}

    \section{What is a digital twin?}
    TODO

    \subsection{Methodology}
    TODO

    \subsection{Result}
    TODO

    \subsection{Validation}
    TODO

    \section{What is digital twin quality?}
    TODO

    \subsection{Methodology}
    TODO

    \subsection{Result}
    TODO

    \subsection{Validation}
    TODO

    \section{What are digital twin quality issues?}
    TODO

    \subsection{Methodology}
    TODO

    \subsection{Result}
    TODO

    \subsection{Validation}
    TODO

    \section{Conclusion}
    \label{section:conclusion}
    TODO

    \bibliography{main}
    \bibliographystyle{plain}

\end{document}