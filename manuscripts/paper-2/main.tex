\documentclass[conference]{IEEEtran}
\usepackage[colorlinks,allcolors=blue]{hyperref}
\usepackage{amsfonts}
\usepackage{graphicx}


\begin{document}

    \title{Modeling and Observing ProtoDigital Twin with ROS2 and Omniverse}
    \author{\IEEEauthorblockN{Georg Hackenberg} 
    \IEEEauthorblockA{School of Electrical and\\ Computer Engineering\\ Georgia Institute of Technology\\ Atlanta,
    Georgia 30332--0250\\ Email: mshell@ece.gatech.edu} 
    \and 
    \IEEEauthorblockN{Alican Tüzün\\ and Montgomery Scott} 
    \IEEEauthorblockA{Starfleet Academy\\ San Francisco, California 966782391\\ Telephone: (800) 555--1212\\ Fax: (888) 555--1212}}
   
    \maketitle

    \begin{abstract}
        Digital twins are becoming more and more important for the efficient and effective development and operation of cyber-physical systems.
        However, digital twins are only useful if they reflect the real-world system accurately enough, i.e.their quality is high enough. 
        This claim entails the question, of what the term quality in the context of digital twins means and how it can be measured. 
        In this article, we present our experience with the quality assurance of a digital twin for an assembly line in the automotive industry.
        We explain our preliminary definition of digital twin quality, which we derive from classical quality models for general software systems. 
        Furthermore, we describe quality issues, which we were able to detect in a digital twin of an assembly line in the automotive industry. \
        Finally, we conclude how to leverage our experience in different contexts and how to generalize the underlying approaches.
    \end{abstract}

    \section{Introduction}\label{section:introduction}
    As new and complex technologies have advanced to emerge, it becomes progressively 
    difficult to understand and define their core concepts. One significant example in the 
    area of digitalization is the concept of digital twins, which has resulted in numerous definitions, that vary 
    from one derivation to another. Because, even though the terms `digital' and `twin' are easy to grasp,  their 
    consolidation develops a new challenging concept, which can be resulted in misunderstanding and misapplication. 
    Therefore, a proto example of digital twin, can significantly improve the understanding 
    of the notion of digital twin, and can help to clear the fuzzy boundaries between similar concepts such as digital shadow and digital model.
    
    The idea of the digital twin originated in 2002 by the Grieves, as an ideal form of product life cycle management. 
    The proposal was to create a co-existed information twin of a real system to gather new information to minimize 
    the system needs and waste, such as material, time, and energy. It should be noted that initially this concept was 
    called the information mirroring model and included all the components of today's digital twins: a real system, 
    a virtual system, a connection between them and an ancillary, the virtual simulation. However, after co-authoring with Vickers in 2010, 
    the term `digital twin' was adopted and Grieves simplified the model by excluding the virtual simulation aspect.
    
    A considerable amount of literature has been published on Digital Twins, however, to date there has been little agreement on the precise definition and application. Numerous authors have considered different 
    definitions, including digital twins as a multi-physics environment, an equivalent to a product,
    a digital copy, a cyber component of a Cyber-Physical System, a controlling and monitoring unit,
    and many more. However, if inspected carefully, most of these definitions focus more on the quality 
    factors or applications of digital twins rather than their notion.
    Therefore, despite the subjective phenomena of the definition of digital twins,
    the definition by the Digital Twin Consortium was considered to be more suitable for the authors. 
    Consortium's definition describes digital twins as virtual representations of real systems with 
    a synchronization attribute, presenting a more abstract and objective perspective.\cite{asf}

    Not only in the literature but digital twins also gained popularity nearly in all branches
    of industry, especially in the manufacturing scene.  
    This resulted in the development of general platforms to construct digital twins. 
    Furthermore, for the visualization of the digital twins, 
    game engines such as Unity and  Unreal Engine have become popular tools, 
    and some of them already offer dedicated digital twin platforms. Also for databases,
    there are ontology-based cloud solutions even with newly defined languages for 
    managing the data associated with digital twins.However,While there are some accessible open-source projects, 
    most of the digital twin platforms today are not free to use and may not be easy to access or grasp. 

    To adresss the challanges associated with the accessability and comprehensibility of the notion of digital twins, main purpose of this paper is not only to provide a fully functional free-to-model proto digital twin, but also give observation results, including the challanges and solutions encountered during the creation of a digital model, digital shadow and digital twin. Goal of the process of modelling a digital twin is to digitalize nearly all the components of the real system and properly twin one of them.
    To make the twinning happen, authors used Omniverse as a simulation and visualization tool, 
    and ROS2(Robot Operating System)  as a sensor and communication middleware. 
    To control the behavior of digitalized artifacts, an HC-SR04 ultrasonic sensor, Raspberry Pi,
    and several software has been used. Also, despite the subjective phenomena of the definition 
    of digital twins, the definition by the Digital Twin Consortium were considered
    to be more suitable for the authors. Consortiums definition describes digital
    twins  as virtual representations of real systems with a synchronization attribute,
    presenting a more abstract and objective perspective which will be a guide for the 
    creation of the proto digital twin.

    \section{Modeling the Proto Digital Twin}\label{section:components}
    Several procedures were executed to demonstrate the notion of a proto-digital twin. 
    Initially, a digital model representing the digitalized form of the real system was developed. 
    Subsequently, relative to the digitalized model, a real system was established. 
    Then a digital shadow was generated to accomplish an automatic unidirectional data 
    transfer from the real to the virtual system. And finally, a digital twin, with 
    bidirectional data flow between the real and the virtual system was created. 

    \subsection*{The Digital Model}\label{section:digital_model}
    To create a digital model of the planned real system, two different software programs were used for two different 
    purposes including Fusion 360, a Computer-Aided Design software, and Blender, a 3D-Graphics software. 

    First, pre-existing digitalized real system components were sourced from the internet, and modified.
     The ones that were not found, such as jumper cables, were modeled in the Fusion 360. 
     Once all the required components were available and reached a needed model fidelity level, 
     detailing was stopped and parts were digitally assembled. Lastly, the assembled digital model 
     was converted to a .fbx file.

    After the conversion occurred, the Blender was used, to convert a .fbx file to a .usd 
    file for import into the simulation environment. However, before the conversion, adjustments 
    were made to the .fbx file to fix issues, such as corrupted visuals and hierarchy problems. 
    Later, the fixed file was saved as a .usd file. 

    Finally, Blender was reopened, and the previously converted .usd file was imported. 
    Further adjustments were made to the file and were saved again as a .usd file.
    Upon completion of this process, the file was prepared for importation into the simulation environment.

    \subsection*{Integrating digital model to Omniverse}\label{section:omniverse}
    The chosen simulation platform for this demonstration was Isaac Sim, which is fully integrated into the NVIDIA Omniverse platform. Isaac Sim was designed specifically for the training and testing of complex robotic systems in realistic virtual environments. Despite the simplicity of the demonstration, which was not a complex robotic system, there were two crucial reasons for the platform selection. First, Isaac Sim offers a ROS2 bridge, which enabled a seamless connection between ROS2 and Omniverse, facilitating relatively easy and smooth data exchange. Second, it leverages RTX Graphics, which delivered high-performance graphics rendering, resulting in an improvement in visual fidelity. However, to use the Omniverse platform, an RTX graphics card is required. Therefore, for this demonstration, the authors used an NVIDIA RTX 2060 graphics card, to run the simulation. 

    First, to initialize the simulation, Isaac Sim was installed on a local computer using the Omniverse launcher. Once the installation was completed, the correct ROS2 bridge was selected.  At this step, it was important not to source the ROS2 Workspace within the same terminal that Isaac Sim was operating, for example with bashrc, to prevent potential compilation errors. 

    After the initialization, the modified .usd file was imported into the Isaac Sim scene. After the import, a visual inspection of the individual parts was then conducted, and needed adjustments were made. Lastly,  the required level of fidelity of the digital model, within the simulation environment was reached, and the step of the construction of the real system was ready.
    \subsection*{Real System}\label{section:real_system}
    First, the components were inserted on the prototype breadboard in accordance with the digital model.
    Next, circuit components were connected with the jumper cables to the appropriate general-purpose input/output (GPIO),
    ground, and power pins of Raspberry Pi 4 (RPI4). 

    Afterward, Ubuntu 20.04 was installed on a micro-SD card, 
    and inserted into the RPI4. Second, the cryptographic network protocol SSH was used to connect the 
    RPI4 to a local computer that already had the same version of Ubuntu. Subsequently, 
    the Robot Operating System 2 (ROS2) version of Foxy was installed on both the RPI 4 and 
    the local computer. Lastly, to access and control the GPIO pins of the RPI4, 
    the Wiring Pi library was installed on RPI4.

    ROS2 was used as a sensor and communication middleware to give behavior to the real system components.
    A custom package was developed to read data from the ultrasonic sensor and control the state of three LEDs. 
    This behavior were integrated into a single node, which was executed with a single-thread executor.
    Lastly, a combined ROS2 node was run, and the state space of the real system was observed to ensure
    that the three LEDs were giving the expected state.

    \subsection*{Digital Shadow}\label{section:digital_shadow}
    
    %WRITE LEAD HERE

    To use the ROS2 bridge within the Isaac Sim, action graphs were utilized. 
    Isaac Sim already had several built-in ROS2 nodes, and authors instead of writing their own ROS2 nodes, 
    used the built-in ones just for the simplification of the demonstration. 
    With the help of these built-in nodes, a visual script has been written, and Isaac Sim subscriebed
    and published the topics that corresponds 1:1 with  the written ROS2 package for the real system.

    Finally, the ROS2 package and simulation environment were initialized 
    and the behavior of the three LEDs was observed to ensure that their state was 
    synchronous and unidirectional between the real and virtual systems. 

    \subsection*{Digital Twin}\label{section:digital_twin}
    The key difference between the digital shadow and digital twin lies within their mode of communication, 
    which the latter is bidirectional instead of the unidirectional behavior like the first one.
    To achieve this bidirectional behavior, action graph was expended and gave a new 
    behavior to the real system: a blue LED  controlled by a counter embedded in the simulation environment. 

    In order to see the effect of the bidirectional behavior, ROS2 node and simulation environment 
    were initialized again, and the behaviour of the three LEDs were monitored to ensure that their state was 
    synchronous and bidirectional.

    \section{Observations}\label{section:results}
    The modeling process of the digital twin with ROS2 and Omniverse showed that modeling the digital twin was relatively easy to construct, 
    debug, and free to model while catching a high degree of visual fidelity. 
    \subsection*{Observations while Creating Digital Model}
    The digital model was created, with several steps,
    which resulted in great quality, which was later used for the digital shadow and digital twin.
    However, during this process, several difficulties occurred and were solved. 

    First, the authors did not consider the CAD environment as a digital environment, 
    instead considered Omniverse as a digital environment. 
    Therefore, in this application, conversion of CAD files to 
    .obj or .fbx and later to .usd was inevitable. The results of the conversions 
    show that a significant amount of model data gets lost, corrupted, or duplicated 
    during the file conversions because of the complexity of the models. 
    Therefore to minimize the data loss, reducing the number of parts
    by grouping parts together as one solid part and removing the unnecessary 
    details, modifying the names of the parts to simple ones, and deleting material
    properties were helpful. However, some of these processes are irreversible, 
    hence they might reduce the fidelity of the model drastically.

    Further observations show that, during the process of the modeling, non-conventional ways,
    such as modeling electric cables as 3D in CAD increased the overall fidelity of the digital model,
    as well as the digital twin significantly. 

    Lastly, there was a significant difference in data loss, between the quality of file converters, 
    Blender and Embedded Omniverse Converter, when the conversion of the .fbx file to a .usd file was wanted. 
    Hence, before putting the .fbx file directly into the omniverse, putting it into the blender and modifying
    there,  gives a better conversion of .fbx to .usd as well as gives more flexibility to manipulate the
    corrupted data which occurs during the conversion to .fbx file from .step file.

    \subsection*{Observations while creating a Real System}
    With the help of the digital model, real system assembly time was reduced 
    significantly as well as overall process safety increased.

    First, the assembly time of the real system according to the digital model, 
    dramatically decreased, due to the provided spatial and visual data from the digital model,
    especially the cabling between the RPI and the breadboard. 

    Also the safety of components, during the assembly, also increased, which prevented
    common mistakes such as giving higher voltage to 
    GPIO pins, which could result in damaging the RPI, was prevented. 

    \subsection*{Observations of manual connection between the real and virtual system}
    \subsection*{Observations while Creating a Digital Shadow}
    \subsection*{Observations of automatic unidirectional connection between the real and virtual system}
    \subsection*{Observations while Creating a Digital Twin}
    \subsection*{Observations of automatic bidirectional connection between the real and virtual system}
    \section{Discussion}\label{section:discussion}
    The last sentence of the introduction should be the first sentence of the discussion.
    How did your actual results compare with what you expected?
    Unexpected results
    Bring everything together, possible applications, extensions for further improvements
    \bibliography{main}
    \bibliographystyle{plain}
    
\end{document}