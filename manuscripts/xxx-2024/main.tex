\documentclass[conference]{IEEEtran}

\usepackage[colorlinks,allcolors=blue]{hyperref}
\usepackage{amsfonts}
\usepackage{graphicx}

\begin{document}

    \title{A formal theory of digital twins}

    \author{
        \IEEEauthorblockN{Georg Hackenberg} 
        \IEEEauthorblockA{School of Engineering\\University of Applied Sciences Upper Austria\\4600 Wels, Austria} 
        \and 
        \IEEEauthorblockN{Alican Tüzün} 
        \IEEEauthorblockA{Josef Ressel Center for Data-Driven Innovation in Business Modeling\\University of Applied Sciences Upper Austria\\4xxx Steyr, Austria}
    }
   
    \maketitle

    \begin{abstract}
        TODO
    \end{abstract}

    \section{Introduction}
    \label{section:introduction}
    TODO~\cite{asf}

    \subsection{Research objective}
    More systematic development of high-quality digital twins in different application domains.

    \subsection{Research question}
    Unambiguous definition of the term "digital twin"?

    \subsection{Research methodology}
    Literature review~\ref{section:related} + formalization~\ref{section:theory} + discussion~\ref{section:discussion}.

    \section{Related work}
    \label{section:related}
    TODO

    \section{A digital twin theory}
    \label{section:theory}
    In the following, we introduce a formal theory of digital twins.
    We start with introducing core concepts in Section~\ref{section:theory-core}.
    Then, on top of the core concepts we build a theory of the real world in Section~\ref{section:theory-universe}.
    Finally, on top of the theory of the real world we derive a theory of digital twins in Section~\ref{section:theory-twin}.

    \subsection{Time, state, and trace}
    \label{section:theory-core}

    \begin{figure}[htbp]
        \centering
        \includegraphics{./figures/theory-core.pdf}
        \caption{Illustration of time, state, and trace.}
        \label{figure:theory-core}
    \end{figure}

    The core concepts include the notion of time, the notion of state, and the notion of behavior.

    \subsubsection{Time}
    In our theory we do not prescribe a specific model of time such as discrete or continuous.
    We rather employ a generic approach, where different models of time can be plugged in easily.
    In this generic approach, time is defined as some (not further specified) set
    \[
        T.
    \]

    \subsubsection{State}
    Similar to the above case, we do not prescribe a specific model of state such as subatomic, atomic, molecular, mechanical, electrical, digital, etc.
    Again we employ a generic approach, where various models of state can be plugged in later.
    In this generic approach state is defined as some (not further specified) set
    \[
        S.
    \]

    \subsubsection{Trace}
    Finally, we include a generic model of how state changes over time.
    Analogous to the previous two cases, concrete models of time and state can be plugged in easily.
    In our approach, a behavior is a mapping from the time domain to the state domain, i.e.
    \[
        \overrightarrow{S}: T \rightarrow S.
    \]

    \subsection{Product, environment, and universe}
    \label{section:theory-universe}

    \begin{figure}[htbp]
        \centering
        \includegraphics{figures/theory-universe.pdf}
        \caption{Illustration of product, environment, and universe.}
        \label{figure:theory-universe}
    \end{figure}

    Based on the concepts of time, state, and behavior we build our model of the real world.
    This model distinguishes between a product and its environment.
    For both the product and its environment we define their states and their functions.
    Finally, we explain the concept of world behavior.

    \subsubsection{Product}
    In our theory, the product state describes the product perfectly at a particular point in time.
    We assume perfect knowledge in this representation and, hence, the representation is only a theoretical construct.
    Mathematically, the product state is defined as set
    \[
        S_{Prod}.
    \]
    TODO
    \[
        \overrightarrow{S_{Pro}}: T \rightarrow S_{Pro}.
    \]

    \subsubsection{Environment}
    Similarly, the environment state describes the environment perfectly at a particular point in time.
    Again we assume perfect knowledge in this purely theoretical representation.
    Hence, the environment state is defined mathematically as set
    \[
        S_{Env}.
    \]
    TODO
    \[
        \overrightarrow{S_{Env}}: T \rightarrow S_{Env}.
    \]

    \subsubsection{Universe}
    TODO
    \[
        S_{Uni} = S_{Pro} \times S_{Env}.
    \]
    TODO
    \[
        \overrightarrow{S_{Uni}}: T \rightarrow S_{Uni}.
    \]
    TODO
    \[
        \overrightarrow{S_{Uni}}: T \rightarrow S_{Pro} \times S_{Env}.
    \]
    such that $\forall t \in T$
    \[
        \overrightarrow{S_{Uni}}(t) = (\overrightarrow{S_{Pro}}(t), \overrightarrow{S_{Env}}(t))
    \]

    \subsection{Function}
    TODO

    \begin{figure}[htbp]
        \centering
        \includegraphics{./figures/theory-world.pdf}
        \caption{Illustration of real world concepts}
        \label{figure:theory-function}
    \end{figure}
    
    \subsubsection{Function}
    Based on the concept of product and environment we can define the product function.
    The product function describes the logic of how the product reacts to the environment state over time.
    This logic can be described as mapping from environment trace to product trace, i.e.
    \[
        F_{Pro}: \overrightarrow{S_{Env}} \rightarrow \overrightarrow{S_{Pro}}
    \]
    Similarly, we can define the environment function.
    The environment function describes the logic of how the environment reacts to the product state over time.
    Hence, this logic can be described as mapping from product trace to environment trace, i.e.
    \[
        F_{Env}: \overrightarrow{S_{Pro}} \rightarrow \overrightarrow{S_{Env}}
    \]

    %\subsubsection{Universe function}
    %Finally, based on product state and function as well as environment state and function we introduce world behavior.
    %In our theory, world behavior is a combination of consistent product and environment behavior.
    %We express world behavior as tuple
    %\[
    %    B_{World} = (\overrightarrow{S_{Pro}}, \overrightarrow{S_{Env}})
    %\]
    %such that the environment behavior $\overrightarrow{S_{Env}}$ is the result of product behavior $\overrightarrow{S_{Prod}}$, i.e.
    %\[
    %    \overrightarrow{S_{Env}} = F_{Env}(\overrightarrow{S_{Pro}})
    %\]
    %and, adversely, the product behavior $\overrightarrow{S_{Prod}}$ is the result of environment behavior $\overrightarrow{S_{Env}}$, i.e.
    %\[
    %    \overrightarrow{S_{Prod}} = F_{Pro}(\overrightarrow{S_{Env}}).
    %\]

    %\subsection{TODO}
    %TODO

    %\subsubsection{Observer}
    %TODO

    %\subsubsection{Observation}
    %TODO

    %\subsubsection{Event}
    %TODO time point and time span (observer dependent)

    \subsection{Our model of digital twins}
    \label{section:theory-twin}
    TODO

    \section{Discussion}
    \label{section:discussion}
    TODO

    \section{Conclusion}
    \label{section:conclusion}
    TODO
    
    \bibliography{main}
    \bibliographystyle{plain}
    
\end{document}