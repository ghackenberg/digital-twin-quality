\documentclass[conference]{IEEEtran}

\usepackage[colorlinks,allcolors=blue]{hyperref}
\usepackage{amsfonts}
\usepackage{graphicx}

\begin{document}

    \title{A formal theory of digital twins}

    \author{
        \IEEEauthorblockN{Georg Hackenberg} 
        \IEEEauthorblockA{School of Engineering\\University of Applied Sciences Upper Austria\\4600 Wels, Austria} 
        \and 
        \IEEEauthorblockN{Alican Tüzün} 
        \IEEEauthorblockA{Josef Ressel Center for Data-Driven Innovation in Business Modeling\\University of Applied Sciences Upper Austria\\4xxx Steyr, Austria}
    }
   
    \maketitle

    \begin{abstract}
        TODO
    \end{abstract}

    \section{Introduction}
    \label{section:introduction}
    TODO~\cite{asf}

    \subsection{Research objective}
    More systematic development of high-quality digital twins in different application domains.

    \subsection{Research question}
    Unambiguous definition of the term "digital twin"?

    \subsection{Research methodology}
    Literature review~\ref{section:related} + formalization~\ref{section:theory} + discussion~\ref{section:discussion}.

    \section{Related work}
    \label{section:related}
    TODO

    \section{A digital twin theory}
    \label{section:theory}
    In the following, we introduce a formal theory of digital twins.
    We start with introducing core concepts in Section~\ref{section:theory-core}.
    Then, on top of the core concepts we build a theory of the real world in Section~\ref{section:theory-world}.
    Finally, on top of the theory of the real world we derive a theory of digital twins in Section~\ref{section:theory-twin}.

    \subsection{Core concepts}
    \label{section:theory-core}
    The core concepts include the notion of time, the notion of state, and the notion of stream.

    \subsubsection{Time}
    In our theory we do not prescribe a specific model of time such as discrete or continuous.
    We rather employ a generic approach, where different models of time can be plugged in easily.
    In this generic approach, time is defined as some (not further specified) set
    \[
        T.
    \]

    \subsubsection{State}
    Similar to the above case, we do not prescribe a specific model of state.
    Again we employ a generic approach, where various models of state can be plugged in later.
    In this generic approach state is defined as some (not further specified) set
    \[
        S.
    \]

    \subsubsection{Stream}
    Finally, we include a generic model of how state changes over time.
    Analogous to the previous two cases, concrete models of time and state can be plugged in easily.
    In our approach, a stream is a mapping from the time domain to the state domain, i.e.
    \[
        \overrightarrow{S}: T \rightarrow S.
    \]

    \subsection{Our model of the real world}
    \label{section:theory-world}
    Based on the concepts of time, state, and stream we build our model of the real world.
    Our model of the real world distinguishes between a product and its environment.
    For both the product and the environment we define the set of possible states as well as a behavior function.
    Finally, we explain what an actual behavior of product and environment is.

    \subsubsection{Environment state}
    \[
        S_{Env}    
    \]

    \subsubsection{Product state}
    \[
        S_{Pro}    
    \]

    \subsubsection{Environment behavior}
    \[
        B_{Env}: \overrightarrow{S_{Prod}} \rightarrow \overrightarrow{S_{Env}}
    \]
    
    \subsubsection{State behavior}
    \[
        B_{Prod}: \overrightarrow{S_{Env}} \rightarrow \overrightarrow{S_{Prod}}
    \]

    \subsubsection{World behavior}
    tuple of product state stream and environment state stream
    \[
        B_{World} = (\overrightarrow{S_{Prod}}, \overrightarrow{S_{Env}})
    \]
    such that state streams are consistent, i.e.
    \[
        B_{Env}(\overrightarrow{S_{Prod}}) = \overrightarrow{S_{Env}}
    \]
    and
    \[
        B_{Prod}(\overrightarrow{S_{Env}}) = \overrightarrow{S_{Prod}}
    \]

    \subsection{Our model of digital twins}
    \label{section:theory-twin}
    TODO

    \section{Discussion}
    \label{section:discussion}
    TODO

    \section{Conclusion}
    \label{section:conclusion}
    TODO
    
    \bibliography{main}
    \bibliographystyle{plain}
    
\end{document}